\documentclass[11pt,twocolumn]{article}

\usepackage{hyperref}
\usepackage{fullpage}

\linespread{1.2}

\begin{document}

\title{CSE 549\\Computational Biology -- Project Proposal\\Genome Compression}
\author{Pragya Pande \and Dhruv Matani}
\date{\today}

\maketitle

\vspace{0.5in}

\section*{The problem}
The Complete Human Genome was sequenced in 2003. Since then a lot of
research is being done in genomics and computational biology. The
major input for most of the computation is the 2.9 billion base pairs
\cite{2}\cite{3} of the human genome which correspond to a maximum of about 
725 megabytes of data human genome.\cite{4}

Furthermore, reduction in the cost of sequencing(via the “next-gen” 
sequencing platforms),has given birth to the 1000 genome project
(1000genomes.org) which aims to sequence the genomes of a large 
number of people. Just like the other human genome reference projects,
this data (estimated 8.2 billion bases per day) would be made
available via public databases for the scientific community.\cite{5}

As we can now see, we are dealing with megabytes and megabytes of data 
when we do work with genomes! This has resulted in rise to challenging 
problems with respect tostorage, distribution (downloading, copying), 
and sharing of this genomic data. Hence we need to consider better 
compression techniques for the genomic data.

In this project, we explore this problem of genome compression and see
if we can compress the human genome to more than what the previous
researcher have done.\cite{6}

\section*{Applicability}


\section*{Current work}

A group of researchers from The University of California at Irvine
lead by Dr. Chen Li\footnote{http://www.ics.uci.edu/~chenli/} have
researched the compressibility of the human genome against a reference
genome and have achieved fascinating results. They have been able to
compress James Watson's genome\footnote{James Watson's genome is a
  418MiB compressed downloadable tarball} which is 3 billion base
pairs long to about 4MB of compressed data. This is small enough to be
emailed across the globe!\cite{xyz}


\subsection*{Handling FASTA files}

\subsection*{}

\section*{The plan}

\subsection*{Getting the data}

\subsection*{Rough estimates of compressibility}

\begin{thebibliography}{9}

\bibitem{xyz}
AbCd
\bibitem{2}
http://www.strategicgenomics.com/Genome/index.htm
\bibitem{3}
http://www.nature.com/nature/journal/v431/n7011/abs/nature03001.html
\bibitem{4}
http://en.wikipedia.org/wiki/Human\_genome
\bibitem{5}
http://www.1000genomes.org/about
\bibitem{6}
http://bioinformatics.oxfordjournals.org/content/25/2/274.full

\end{thebibliography}



\end{document}
